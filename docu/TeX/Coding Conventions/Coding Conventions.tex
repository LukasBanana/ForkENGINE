\documentclass{article}
\title{ForkENGINE Coding Conventions}
\author{Lukas Hermanns}
\date{March 2014}

\usepackage{listings}
\usepackage{color}
\usepackage{pxfonts}
\usepackage{geometry}
\usepackage[T1]{fontenc}

\geometry{
	a4paper,
	top=20mm,
	bottom=20mm
}

\begin{document}

\definecolor{brightBlueColor}{rgb}{0.5, 0.5, 1.0}
\definecolor{darkBlueColor}{rgb}{0.0, 0.0, 0.5}

\lstset{
	language = C++,
	basicstyle = \footnotesize\ttfamily,
	commentstyle = \itshape\color{brightBlueColor},
	keywordstyle = \bfseries\color{darkBlueColor},
	stringstyle = \color{red},
	frame = single,
	tabsize = 2
}

\maketitle
\begin{itemize}

\item
\textbf{Namespaces}, \textbf{classes}, \textbf{structures}, \textbf{enumerations}, \textbf{enumeration entries}, \textbf{functions},
\textbf{function objects} (also lambdas), \textbf{function interfaces} and \textbf{template typename or class arguments} begin in \textit{upper case}:

\begin{lstlisting}
namespace Math
{
class Vector2 { /* ... */ };
}

enum class ColorFlags
{
	Red, Green, Blue
};

auto AbsDotProduct = [](const Math::Vector2f& a, const Math::Vector2f& b)
{
	return
		std::abs(a.x)*std::abs(b.x) +
		std::abs(a.y)*std::abs(b.y);
};

typedef std::function<void (int x, int y)> DrawCallback;

template <typename Type, class Class, int num, ColorFlags flag> /* ... */
\end{lstlisting}

\item
\textbf{Variables}, \textbf{constants} and \textbf{static-constants} begin in \textit{lower case}:

\begin{lstlisting}
std::string firstName;
std::string filename;
std::vector<Vertex> vertices;
Video::VertexBufferPtr vertexBuffer;
static const float pi = 3.141592654f;
\end{lstlisting}

\item
\textbf{Macros} are always entirely in \textit{upper case}. Names are separated by \textit{underscores}:

\begin{lstlisting}
#define MAX_NUM_LIGHTS  8
#define LIGHT_MIN       0.2
#define LIGHT_RANGE     (1.0 - LIGHT_MIN)
\end{lstlisting}

\item
Curly brackets are written like it's typical for C++:

\begin{lstlisting}
// Examples:
void Function()
{
	/* Some statements */
}

if (/* Any expression */)
{
	/* Some condition statements */
}

{
	/* Simple block */
}
\end{lstlisting}

Don't write Java like `Egyptian Brackets':

\begin{lstlisting}
// False examples:
void Function() {
	/* ... */
}

if (/* ... */) {
	/* ... */
}
\end{lstlisting}

An exception for this is when you write very small lambdas:

\begin{lstlisting}
auto IsWhiteSpace = [](char chr) { return chr == ' ' || chr == '\t'; };
\end{lstlisting}

\item
All \textbf{private} member variables end with underscore ('\_'), e.g. "int memberVar\_;", "Math::Vector3f position\_;" etc.

\item
Mixed public and private (or protected) class members are allowed, when they are meaningful.

\begin{lstlisting}
class Person
{
	public:
		Person(int age);
		int GetAge() const;
		std::string name;
	private:
		int age_;
};
\end{lstlisting}

\item
Every simple getter function should start with "Get", and every simple setter function should start with "Set".

Example for getter/ setter:

\begin{lstlisting}
float GetHeight() const
{
	return height_;
}
void SetHeight(float height)
{
	height_ = height;
}
\end{lstlisting}

Example for non-getter/ setter: (because the return value is not a simple member variable,
e.g. for a linked list this could take several computations to determine the list size)

\begin{lstlisting}
size_t Size() const
{
	return container.size();
}
void Resize(size_t size)
{
	container.resize(size);
}
\end{lstlisting}

An exception for getters is when static members are returned:

\begin{lstlisting}
class Mouse
{
	public:
		static Mouse* Instance()
		{
			return instance_;
		}
	private:
		static Mouse* instance_;
};
\end{lstlisting}
\begin{lstlisting}
class RenderContext
{
	public:
		static RenderContext* Active()
		{
			return active_;
		}
	private:
		static RenderContext* active_;
};
\end{lstlisting}

\item
\textbf{Hex-literals} are always in \textit{lower case}, e.g. 0xff and not 0xFF.

\item
Templates use "typename" types if its meant to be used for data types (like int, float etc.).

\item
Templates use "class" types if its meant to be used for classes.

\item
\textbf{Custom exceptions} are only in the "Fork" main namespace.

\item
\textbf{Setter} or \textbf{getter} with a "Description" (e.g. "FontDescription") are named "Get/Set...Desc" and not "Get/Set...Description" to keep the name short but clear.

\item
For MS/Windows specific code, don't write "hWnd", "hFont". Instead write "windowHandle", "fontHandle" etc., i.e. use clear names and not weird Microsoft notations.

\item
Make use of the C++11 scoped enumerations:

\begin{lstlisting}
enum class MyEnum
{
	Entry1,
	Entry2,
	/* ... */
};
\end{lstlisting}

\item
Flags enumerations should be written like this:

\begin{lstlisting}
struct MyFlags
{
	typedef unsigned int DataType;
	enum : DataType
	{
		Flag1 = (1 << 0),
		Flag2 = (1 << 1),
		/* ... */
	};
};

MyFlags::DataType flags = MyFlags::Flag1 | MyFlags::Flag2;
\end{lstlisting}

This is due to the strong typing with the C++11's enum classes:

\begin{lstlisting}
// False example:
enum class MyFlags
{
	Flag1 = (1 << 0),
	Flag2 = (1 << 1),
	/* ... */
};

MyFlags flags = MyFlags::Flag1 | MyFlags::Flag2; // <-- Error
\end{lstlisting}

\item
Error messages, telling that something failed, should be written as "[Do-]ing ... failed" and not "Could not [Do]":

\begin{lstlisting}
// Usage example:
IO::Log::Error("Loading texture \"" + filename + "\" failed");

// False example:
IO::Log::Error("Could not load texture \"" + filename + "\"");
\end{lstlisting}
  
\item
Event handler callbacks should never be implemented via function objects. Use class inheritance instead.

\begin{lstlisting}
// Example:
class Widget
{
	public:
		// This is the event handler for all "Widget" events.
		class EventHandler
		{
			public:
				virtual ~EventHandler()
				{
				}
				virtual void OnResize()
				{
				}
			protected:
				EventHandler() = default;
		};
		
		typedef std::shared_ptr<EventHandler> EventHandlerPtr;
		
		void AddEventHandler(const EventHandlerPtr& eventHandler)
		{
			if (eventHandler)
				eventHandlers_.push_back(eventHandler);
		}
		
	private:
		std::vector<EventHandlerPtr> eventHandlers_;
};
\end{lstlisting}

Only when a single small function is required, use a function object:

\begin{lstlisting}
// Example:
class SceneNodePrinter
{
	public:
		// Callback interface
		typedef std::function<void (const std::string& text)> SimpleCallback;
		
		void PrintSceenNode(
			Scene::SceneNode* sceneNode, const SimpleCallback& callback
		);
		
	private:
		SimpleCallback callback_;
};

SceneNodePrinter printer;

printer.PrintSceneNode(
	sceneNode,
	[](const std::string& text)
	{
		IO::Log::Message(text)
	}
);
\end{lstlisting}

\item
Never use "NULL"! Instead use "nullptr", except for handles (WinAPI); use "0" for handles, because handles are not always typedefs to pointers.

\item
Avoid too much implicit programming, e.g. the core classes like "Vector3" only have explicit constructors, when only one input parameter is used:

\begin{lstlisting}
explicit Vector2(float size);
explicit Vector3(float size);
explicit Vector4(float size);
\end{lstlisting}

\item
Boolean class members should start with "is", "has", "was" etc. telling what kind of boolean this is.

\begin{lstlisting}
bool isEnabled;
bool hasChildren;
bool wasMovedLastFrame;
bool acceptModifications;
bool preventForPowerSafe;
\end{lstlisting}

\item
In a class hierarchy, when up-casting is required, the base class should have the following function interface:

\begin{lstlisting}
class BaseClass
{
	public:
		enum class Types
		{
			Type0,
			Type1,
			//...
		};
		virtual Types Type() const = 0;
};
\end{lstlisting}

Each derivated class must then implement the function "Type", which returns the respective class type.

Here is an example class:

\begin{lstlisting}
// Base class
class Texture
{
	public:
		enum class Types
		{
			Tex1D,
			Tex2D,
			Tex3D,
			TexCube,
		};
		virtual Types Type() const = 0;
		//...
};

// Sub class
class Texture1D : public Texture
{
	public:
		Types Type() const
		{
			return Types::Tex1D;
		}
		//...
};
\end{lstlisting}

\item
Special enumeration entries are written with two clasping underscores:

\begin{lstlisting}
enum ComponentTypes
{
	// Special entry for functions which always return an enumeration item.
	__Unknown__,
	
	Asset,
	Entity,
	
	// Special entry which is used to determine the number of items.
	__Num__ = 2,
};
\end{lstlisting}

\item
`Enumeration entry to string' conversion functions are always \textit{static functions} and
called \textbf{"<EnumName>ToString"} where \textit{<EnumName>} specifies the singular enumeration name, e.g.:

\begin{lstlisting}
class Entity
{
	enum Types
	{
		Model,
		Camera,
		Light,
	};
	static std::string TypeToString(const Types type);
	
	enum Modes
	{
		Editing,
		Moving,
	};
	static std::string ModeToString(const Modes mode);
};
\end{lstlisting}

\item
\textbf{Code documentation} is written in \textit{doxygen} syntax.
Keep the same order of documentation entries (i.e. parameters, return type, remarks, etc.):

\begin{lstlisting}
/**
This is any function.
\param[in] x Specifies the X coordinate.
\param[in,out] y Specifies the Y coordinate.
\return Raw pointer to an integer array.
\remarks This is some additional and more detailed information.
\note Some warning information, e.g: make sure to delete the integer pointer!
\todo Some todo information.
\throws InvalidArgumentException If 'x' is negative.
\see SomeOtherFunction
*/
int* AnyFunction(int x, int& y);

//! Single line documentation.
void SmallFunction();
\end{lstlisting}

\item
\textbf{Singleton} classes are implemented as follows:

\begin{lstlisting}
class SingletonExample
{

	public:
		
		SingletonExample(const SingletonExample&) = delete;
		SingletonExample& operator = (const SingletonExample&) = delete;
		
		static SingletonExample* Instance()
		{
			static SingletonExample instance;
			return &instance;
		}
		
	private:
	
		SingletonExample() = default;
		
};
\end{lstlisting}

\item
\textbf{Ranges} in documentation are written as follows:

\begin{lstlisting}
/**
\param[in] index Range is [0 .. number-of-indices).
\param[in] interpolator Range is [0.0 .. 1.0].
\param[in] anyFloat Range is (-inf .. +inf).
*/
void ExampleFunction(size_t index, float interpolator, float anyFloat);
\end{lstlisting}

\item
\textbf{Identification numbers} are written in short as "ID" and not "Id" or "id".

\end{itemize}

\end{document}