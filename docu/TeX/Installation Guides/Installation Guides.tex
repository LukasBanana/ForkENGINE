\documentclass{article}
\title{ForkENGINE Installation Guides}
\author{Lukas Hermanns}
\date{March 2014}

\usepackage{listings}
\usepackage{color}
\usepackage{pxfonts}
\usepackage{geometry}
\usepackage{hyperref}

\geometry{
	a4paper,
	top=20mm,
	bottom=20mm
}

\begin{document}

\definecolor{brightBlueColor}{rgb}{0.5, 0.5, 1.0}
\definecolor{darkBlueColor}{rgb}{0.0, 0.0, 0.5}

\lstset{
	language = C++,
	basicstyle = \footnotesize\ttfamily,
	commentstyle = \itshape\color{brightBlueColor},
	keywordstyle = \bfseries\color{darkBlueColor},
	stringstyle = \color{red},
	frame = single,
	tabsize = 2
}

\maketitle

\section{Compiler}
The ForkENGINE has been written in modern C++11. Currently the following compilers are supported:

\begin{itemize}

\item
\href{http://www.microsoft.com/en-us/download/details.aspx?id=40787}{Microsoft VisualC++ 2013 (MSVC12)}

\item
\href{http://www.mingw.org/}{MinGW 4.7.1 (GNU C++)}

\end{itemize}

\section{Setup}

\subsection{Include Directory}
Set the include directory to "ForkENGINE/include".
In your sources you can then include the engine modules like in the following example:

\begin{lstlisting}
#include <fengine/core.h>
#include <fengine/video.h>
#include <fengine/scene.h>
\end{lstlisting}

Or if you always want to include everything, use this header file:

\begin{lstlisting}
#include <fengine/import.h>
\end{lstlisting}

\subsection{Library Directory}
There are several library directories.

\begin{itemize}

\item
Use "ForkENGINE/lib/MSVC12/Win32" when you are creating a 32-bit project with Microsoft Visual Studio 2013.

\item
Use "ForkENGINE/lib/MSVC12/Win64" when you are creating a 64-bit project with Microsoft Visual Studio 2013.

\item
Use "ForkENGINE/lib/MinGW" when you are creating a 32-bit project with MinGW on MS/Windows.

\item
Use "ForkENGINE/lib/Posix" when you are creating a 32-bit project with GNU/C++ on a posix system (e.g. GNU/Linux).

\end{itemize}

Only link to the libraries you need in your projects.
You'll find a list of all libraries and a brief description in the "Libraries" section.

\subsection{Binary Directory}

\subsubsection{MS/Windows}
On MS/Windows, the engine consists of the following binary files:

\begin{itemize}

\item
ForkAnimation.dll

\item
ForkAudio.dll

\item
ForkAudioAL.dll (\textit{OpenAL} audio system)

\item
ForkAudioXA2.dll (\textit{XAudio 2} audio system)

\item
ForkCore.dll (\textit{IO}, \textit{Math} and \textit{Platform} core systems)

\item
ForkENGINE.dll

\item
ForkNetwork.dll

\item
ForkPhysics.dll

\item
ForkPhysicsNw.dll (\textit{Newton Game Dynamics} physics system)

\item
ForkPhysicsPx.dll (\textit{NVIDIA PhysX} physics system)

\item
ForkRenderer.dll

\item
ForkRendererD3D11.dll (\textit{Direct3D 11.0} render system)

\item
ForkRendererGL.dll (\textit{OpenGL} render system)

\item
ForkScript.dll

\item
ForkScene.dll

\item
ForkUtility.dll

\item
OpenAL32.dll (OpenAL low level audio library)

\end{itemize}

The best way to use the binary files is to add the directory (respective to the used library directory,
(e.g. "ForkENGINE/bin/MSVC12/Win32") to the PATH variable on MS/Windows.

\subsubsection{Posix}
On posix systems (e.g. GNU/Linux), the engine consists of the following binary files:

\begin{itemize}

\item
ForkAnimation.so

\item
ForkAudio.so

\item
ForkAudioAL.so (\textit{OpenAL} audio system)

\item
ForkCore.so

\item
ForkENGINE.so

\item
ForkNetwork.so

\item
ForkPhysics.so

\item
ForkPhysicsNw.so (\textit{Newton Game Dynamics} physics system)

\item
ForkRenderer.so

\item
ForkRendererGL.so (\textit{OpenGL} render system)

\item
ForkScript.so

\item
ForkScene.so

\item
ForkUtility.so

\end{itemize}

\subsection{Dependencies}
The ForkENGINE has been written very modular, i.e. some libraries are loaded dynamically during run-time.
The render system implementations for instance are not statically linked to the engine and therefore
should not be linked statically to your projects.

The advantage is that if the host computer does not provide some libraries,
your application can handle this dynamically and choose another render system.

Consider the following situation: your application wants to create a Direct3D 11 render system
but the host computer does not have the d3d11.dll on MS/Windows.
A common result is, that the user gets an error message which states,
that a DLL is missing and the program should be re-installed.

In the case of the ForkENGINE your application can catch an exception and
create another render system, OpenGL for instance (which runs almost on all platforms).
This will reduce errors and libraries can be added or removed in a dynamic manner.

\section{Libraries}

The ForkENGINE consists of the following libraries:

\subsection{ForkCore}
Contains core functionalities: platform dependent code (such as creating a frame - also called "window"),
input devices (mouse and keyboard), extended file access and math functions.

\subsection{ForkScene}
Scene management library.

\subsection{ForkAnimation}
Animation library.

\subsection{ForkRenderer}
Render system library.
There are further libraries which will be loaded dynamically during run-time.
These are "ForkRendererGL" (OpenGL) and "ForkRendererD3D11" (Direct3D 11.0).

\subsection{ForkNetwork}
Network library.

\subsection{ForkUtility}
Utility library.

\subsection{ForkAudio}
Audio library.
There are further libraries which will be loaded dynamically during run-time.
These are "ForkAudioAL" (OpenAL) and "ForkAudioXA2" (XAudio 2).

\subsection{ForkPhysics}
Physics and collision library.
There are further libraries which will be loaded dynamically during run-time.
These are "ForkPhysicsNw" (Newton Game Dynamics) and "ForkPhysicsPx" (NVIDIA PhysX).

\subsection{ForkScript}
Scripting library. This includes the \textbf{Mono scripting engine}.

\subsection{ForkENGINE}
Engine Device etc., connects all sub-libraries.

\end{document}